\documentclass{article}
\usepackage{graphicx} % Required for inserting images
\usepackage[utf8]{inputenc}
\usepackage[T2A]{fontenc}
\usepackage[russian]{babel}
\usepackage{amsfonts}
\usepackage{amsmath}
\usepackage{amssymb}
\usepackage{fancyhdr}
\usepackage{float}
\usepackage[left=3cm,right=3cm,top=3cm,bottom=3cm]{geometry}
\usepackage{graphicx}
\usepackage{hyperref}
\usepackage{indentfirst}
\usepackage{multicol}
\usepackage{stackrel}
\usepackage{xcolor}
\author{Бугрий Илья M3134}
\date{November 2023}

\begin{document}
\begin{enumerate}
    \item Необходимо доказать, что для любого вектора линейного пространства $X(K)$
    $$
        \sum\limits_{i}{\alpha_i f^{i}} = \Theta \Longleftrightarrow \forall \alpha_i = 0 \Longleftrightarrow \nexists \{\alpha_i\}^{n}_{i = 1}: \forall x \in X(K) \sum\limits_{i}{\alpha_i (f^{i}, x)} = 0;
    $$
    \item Любой вектор $x \in X(K)$ можно разложить по базису $\implies \sum\limits_{i}{\alpha_i (f^{i}, x)} = \sum\limits_{i}{\alpha_i(f^{i}(\xi_1 e_1) + \dots + f^{i}(\xi_n e_n))} = \\
    = \sum\limits_{i}{\alpha_i f^{i}(\xi_1 e_1) + \dots + \alpha_i f^{i}(\xi_n e_n)}$ 
    \item Нужно доказать, что $e_j: \sum\limits_{i} \alpha_i (f^i, e_j) = 0 \Longleftrightarrow \alpha_j = 0$
    \item $\sum\limits_{i} \alpha_i (f^i, e_j) = \sum\limits_{i} \alpha_i \delta^{i}_{j} \implies$ это равно нулю только когда $\alpha_j = 0$
\end{enumerate}
\end{document}