
\documentclass{article}
\usepackage{graphicx} % Required for inserting images
\usepackage[utf8]{inputenc}
\usepackage[T2A]{fontenc}
\usepackage[russian]{babel}
\usepackage{amsfonts}
\usepackage{amsmath}
\usepackage{amssymb}
\usepackage{fancyhdr}
\usepackage{float}
\usepackage[left=3cm,right=3cm,top=3cm,bottom=3cm]{geometry}
\usepackage{graphicx}
\usepackage{hyperref}
\usepackage{indentfirst}
\usepackage{multicol}
\usepackage{stackrel}
\usepackage{xcolor}
\usepackage{listings}
\author{Бугрий Илья M3134}
\date{November 2023}

\begin{document}
\begin{enumerate}
    \item [9.149]
    \begin{itemize}
        \item $w_i$ - произвольная перестановка
        \item $y(w_i)$ - количество неподвижных точек в ней
        \item $p(w_i) = \frac{1}{n!}$ - вероятность случайной перестановки
        \item $E(y) = \sum\limits_{i}{y(w_i)p(w_i)} = \frac{1}{n!}\sum\limits_{i}{y(w_i)}$
        \item Будем считать общее количество неподвижных точек во всех перестановках по рядам.
        \item Мы нарисуем двудольный граф, где слева все точки, а справа все перестановки, где ребро есть между точкой и перестановкой только, \\
        когда точка неподвижна в данной перестановке. Количество ребер равно необходимому количеству. Кол-во ребер можно посчитать используя только левые точки.
        \item Зафиксируем $a_i$, возможных перестановок, где $a_i$ будет неподвижной точкой $(n-1)!$, включая те, где больше одной неподвижной точки.
        \item $n (n-1)!$ - необходимое количество $\implies E(y) = 1$
    \end{itemize}

    \item [10.152] 
    \begin{itemize}
        \item 
        \begin{lstlisting}
            for i in range(n):
                for j in range(k):
                    dp[i + j + 1] = min(dp[i + j + 1], dp[i] + a[i + j + 1])
        \end{lstlisting}
    \end{itemize}

    \item [10.153]
    \begin{itemize}
        \item На i-том шаге поддерживаем очередь на минимум из предыдущих k элементов 
        \item Очередь на двух стеках
        \begin{itemize}
            \item Элемент очереди: (элемент, минимальный элемент в этом стеке среди элементов добавленных ранее)
            \item Добавление: (элемент, q1.front || элемент)
        \end{itemize}
        \item При переходе добавляем элемент в очередь и удаляем q.front
        \item Очередь работает за $\mathcal{O}(1)$ так как каждый элемент добавляется и удаляется только 1 раз, в итоге 3n операций
    \end{itemize}

    \item [10.154] 
    \begin{itemize}
        \item 1. $dp[i - 1] > dp[i - 2] \implies dp_{path}[i] = dp_{path}[i-2]$
        \item 2. $dp[i - 1] < dp[i - 2] \implies dp_{path}[i] = dp_{path}[i - 1]$
        \item 3. $dp[i-1] == dp[i-2] \implies dp_{path}[i] = dp_{path}[i-1] + dp_{path}[i-2]$
    \end{itemize}

    \item [10.155] 
    \begin{itemize}
        \item 1. $dp[i - 1] > dp[i - 2] \implies dp_{path}[i] = dp_{path}[i-2]$
        \item 2. $dp[i - 1] < dp[i - 2] \implies dp_{path}[i] = dp_{path}[i - 1]$
        \item 3. $dp[i-1] == dp[i-2] \implies dp_{path}[i] = dp_{path}[i-1] + dp_{path}[i-2]$
    \end{itemize}

    \item [10.156]
    \begin{itemize}
        \item На i-том шаге ищем лексиграфически минимальный префикс среди $[0, i-1]$
        \item $dp[i] = findMinPrefix(0, i - 1) + a_i$
    \end{itemize}

    \item [10.157]
    \begin{itemize}
        \item Поддерживаем $dp_{even}, dp_{odd}$, каждый элемент которых представляет максимальную сумму в данном элементе.
        \item $a_i$ - odd $\implies$ 
        \begin{align*}
             &dp_{even}[i][j] = max(dp_{odd}[i-1][j], dp_{odd}[i][j- 1]) + a_i \\
             &dp_{odd}[i][j] = max(dp_{even}[i-1][j], dp_{even}[i][j-1]) + a_i \\ 
        \end{align*}
        \item Аналогично с $a_i$ - even
        \item Сумма максимальна так как мы всегда прибавляем к максимальной 
    \end{itemize}

    \item [10.158]
    \begin{itemize}
        \item $dp[i][j][k]$ - количество путей в (i, j) со стоимостью k
        \item щ
        \begin{lstlisting}
            for i in range(n):
                for j in range(m):
                    for l in range(k):
                        if a[i][j] + l > k:
                            cnt += dp[i - 1][j][l]
                            cnt += dp[i][j - 1][l]
                        else:
                            dp[i][j][a[i][j] + l] += dp[i - 1][j][l]
                            dp[i][j][a[i][j] + l] += dp[i][j - 1][l]
        \end{lstlisting}
    \end{itemize}
\end{enumerate}
\end{document}